\subsection{Définitions}
\subsubsection{Langage}  
\begin{itemize}
   \item expression d'une pensée 
   \item faculté / capacité 
   \item ne fait pas forcément appel à la parole
   \item système de signes (vocaux, graphiques, ou numériques)
   \item propre à l'humain
   \item fonction symbolique
   \item parole, écriture
   \item la langue est une forme commune à un même groupe
   \item technique corporelle complexe (langage des abeilles)
   \item principes :
      double-articulation (phonèmes + morphèmes), arbitrarité du signe linguistique (;;;)
\end{itemize}

   définition :
   Le langage est compris comme moyen particulier de communiquer, exprimer ses pensées, ses intentions. (propre à l'homme)



\subsubsection{Communication}
\begin{itemize}
   \item Rapport entre émetteur et récepteur
   \item échange verbal
   \item transmission d'un message
   \item échange de signes
   \item transformation d'un message concret en un système de signes
   \item partage d'une convention 
   \item action de communiquer
   \item relation dynamique
   \item réponse attendue
\end{itemize}

   définition :
   La communication est un moyen de transmettre voire, d'échanger des informations à l'aide d'un code plus ou moins sophistiqué, qui peut être gestuel et/ou verbal.

\subsection{Pour le prochain cours :}
lecture  du texte de journey :
choisir une notrion et/ou auteur et la.e présenter
(peut-être une présentation orale)
