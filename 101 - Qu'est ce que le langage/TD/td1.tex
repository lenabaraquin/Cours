\subsecion{Définitions}
\subsubsection{Langage}  
\begin{itemize}
   - expression d'une pensée 
   - faculté / capacité 
   - ne fait pas forcément appel à la parole
   - système de signes (vocaux, graphiques, ou numériques)
   - propre à l'humain
   - fonction symbolique
   - parole, écriture
   - la langue est une forme commune à un même groupe
   - technique corporelle complexe (langage des abeilles)
   - principes :
      double-articulation (phonèmes + morphèmes), arbitrarité du signe linguistique (###)
\end{itemize}

   définition :
   Le langage est compris comme moyen particulier de communiquer, exprimer ses pensées, ses intentions. (propre à l'homme)



\subsubsection{Communication}
\begin{itemize}
   - Rapport entre émetteur et récepteur
   - échange verbal
   - transmission d'un message
   - échange de signes
   - transformation d'un message concret en un système de signes
   - partage d'une convention 
   - action de communiquer
   - relation dynamique
   - réponse attendue
\end{itemize}

   définition :
   La communication est un moyen de transmettre voire, d'échanger des informations à l'aide d'un code plus ou moins sophistiqué, qui peut être gestuel et/ou verbal.

\subsection{Pour le prochain cours :}
lecture  du texte de journey :
choisir une notrion et/ou auteur et la.e présenter
(peut-être une présentation orale)
