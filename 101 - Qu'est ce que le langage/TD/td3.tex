\subsection{Caractéristiques du signe :}
\begin{itemize}
   \item[A]
      \begin{itemize} 
         \item \textbf{Les signes en question sont non-intentionnels, constituant des sortes de signes naturels qui sont utilisés pour inférer l'existence de quelque chose.}
         \item Naturel 
         \item De l'ordre du mouvment
         \item non intentionnel
      \end{itemize}
   \item[B]
      \begin{itemize} 
         \item \textbf{Regroupe des définitions dans lesquelles les signes peuvent au contraire être qualifiés d'artificiels, dans le sens où ils sont utilisés par des êtres humains pour communiquer sur la base de conventions culturellement établies.}
         \item Artificiel
         \item Représentation artificielle et intentionnelle
      \end{itemize}
   \item[C]
      \begin{itemize} 
         \item \textbf{Regroupe quant à elle des définitions plus étroites, littéraires ou obsolètes, dont on pourrait en dernier ressort des deux valeurs fondamentales identifiées en A et B.}
         \item de l'ordre de l'interprétation
         \item signe naturel interprétés et devient artificiel
      \end{itemize}
\end{itemize}
\\

\subsection{Intentions et Attentions dans la communication}
\textit{fig. 10 du cours}

\begin{enumerate}
         \item test en magasin / situation normale
         \item signe de la tête (avec ambiguïtée d'interprétation par le destinataire)
         \item cours sans attention de l'élève
         \item le locuteur parle sans que l'interlocuteur ne s'en aperçoive 
         \item soupir mal interprété 
         \item entretien d'embauche (paraverbal)
         \item mystère 
\end{enumerate}
\textit{Autres exemples sur IRIS}

\subsection{Point de cours sur les \emph{inférences}}
\textbf{La sous interprétation}\\
\enquote{- Il y a des courants d'air ...}\\
\enquote{- Ah oui, c'est vraiment pénible}\\
   E a froid et attends de D qu'il ferme la porte
\textbf{L'interprétation érronée}\\
\enquote{- Tu ne mets pas tes lunettes?}\\
\enquote{- Si, tu as raison.}\\
   D n'a pas compris que E demande les lunettes de soleil de D
\textbf{Variations culturelles sur les inférences}\\
\enquote{- Vous avez mangé?}\\
\enquote{- Non.}\\
\enquote{- Vous devez avoir faim alors, je vous laisse aller manger.}\\
   E s'attend à ce que D lui propose d'aller manger avec lui


