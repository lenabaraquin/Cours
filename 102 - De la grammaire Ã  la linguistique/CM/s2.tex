% cours.tex s1.tex s2.tex makefile
\section{Extraits à lire}

\subsection{Extrait 1 : Gary-Prieur (1985), \textit{à propos des différences entre la grammaire et la linguistique}}

\textbf{Grammaire :}
    \begin{itemize} 
       \item La première grammaire connue est une description du sanskrit de Panini (grammairien de l'Inde antique, 4\up{ème} siècle avant J.-C.)
       \item Elle a une vocation pédagogique.
       \item Présente une perspective normative : enseigne le bon usage et sanctionne les fautes.
       \item Présente l'idée que le langage reflète la pensée 
    \end{itemize}

\textbf{Linguistique :}
    \begin{itemize} 
         \item \enquote{Le nom de linguistique apparaît pour désigner une démarche spécifique de la grammaire}
         \item \enquote{La linguistique naît dans un domaine qui était traditionnellement celui de la grammaire}
         \item Une discipline scientifique dont l'objectif est la description des langues qui sont des objets de connaissance
         \item À une vocation descriptive
    \end{itemize}
\subsection{F. de Saussure (1972), \textit{à propos des différences entre langage et langue}}

\textbf{Langue :}
\begin{itemize}
   \item produit social de la faculté de langage
   \item ensemble de conventions adoptées par le corps social
   \item acquis
   \item \textit{jusqu'a 17 mois, le bébé est capable de discriminer tous les phonèmes que l'humain peut produire}
\end{itemize}

\textbf{Langage :}
\begin{itemize}
   \item Faculté cognitive innée propre à l'homme
\end{itemize}

\newpage
\part{Chapitre 1}
\section*{Plan du chapitre}
\begin{itemize}
   \item \textit{Cours de linguistique générale}
   \item Qu'est-ce que le langage ?
   \item L'arbitraire du signe linguistique
   \item La non fixité
\end{itemize}

\section{\textit{Cours de linguistique générale}}
Il s'agit d'une publication posthume d'un cours de linguistique générale donné par Louis Ferdinand de Saussure et rédigé à partir des notes de ses étudiants. (Rédaction par C. Bally et A. Sechehaye)
%référence biblio cours de linguistique générale
% et référence biblio de l'édition critique

\subsection{Une œuvre bien particulière}
\begin{quote}
      Si le Cours peut être considéré
comme l'œuvre de Ferdinand de
Saussure, c'est en tout cas comme
une œuvre bien particulière. Cette
particularité s'enracine dans la vision
et dans la volonté de Bally et
Sechehaye. Ceux-ci, quelques
semaines après la mort de Saussure,
après avoir consulté des notes
d'étudiants et quelques autographes
du linguiste disparu, vont, d'une
part, imaginer un livre et, d'autre
part, infléchir le contenu de ce
livre...
\end{quote}
Extrait de : \\
Bouquet, S. (1999). La linguistique
générale de Ferdinand de Saussure :
textes et retour aux textes. Texto !
décembre 1999 [en ligne]
%mettre au propre dans la biblio

\subsection{Une publication incontournable}
Première fois que la recherche sur le langage et les langues tente de :
\begin{itemize}
   \item penser rigoureusement les propriétés de son objet
   \item fixer les limites de son champ
\end{itemize}

\subsection{objectifs du cours}
\begin{itemize}
   \item Nous n'étudierons pas le \textit{Cours de linguistique générale} (même s'il peut être intéressant de le lire).
   \item Mais nous nous intéresserons à quelques unes des propriétés qui y sont décrites :
      \begin{itemize} 
               \item pour montrer leur pertinence 
               \item et pour en faire une sorte d'ossature pour notre cours
      \end{itemize}
   \item Nous nous intéresserons en particulier à deux propriétés : l'arbitraire du signe et la non-fixité
\end{itemize}

\section{Une distinction importante : en mention ou en usage}

Un mot est dit "en usage" lorsqu'il est utilisé pour revoyer à un objet extérieur à lui-même (dans un cas on parle du monde).
\textit{Exemple : je ne mange \textbf{jamais} de blé}
Les mots en usage sont les mots \textbf{avec lesquels} on parle.

Un mot est dit "en mention" lorsqu'il est utilisé pour ne renvoyer qu'à lui-même (dans un cas on parle de mots).
\textit{Exemple: \textbf{"jamais"} est un adverbe.}
Il s'agit alors d'un usage métalinguistique.
Les mots en mention sont les mots \textbf{dont} on parle.
Pour les examens, on souligne ou met entre guillemet les mots en mention.

\subsection{une illustration de l'opposition en usage / en mention}
\textbf{Contexte}
quelques jours après une fusillade
dans une église méthodiste noire de
Charleston en Caroline du Sud, le
président américain Barack Obama
dénonce la persistance du racisme
au États-Unis, dans une interview
diffusée le 22 juin 2015.
Il utilise le mot nigger...

\textit{Extrait d’un billet de Bling (Blog de
linguistique illustré) intitulé « Mots
interdits »}
\textit{http://bling.hypotheses.org/}
%insérer fig1.png

\section{Une langue : l'objet intime et familier à tout locuteur}
\begin{itemize}
   \item Tout locuteur a une expérience quotidienne de sa langue : il sait parler.
   \item Mais il ignore l'histoire de sa langue et les règles qu'il applique pour parler.
   \item Cependant tout locuteur applique les règles de sa langue en permanence.
\end{itemize}

\emph{Extrait du \textit{Cours de linguistique générale}}
\begin{itemize}
   \item \enquote{Ceux-là mêmes qui en font un usage journalier [de la langue] l'ignorent profondément}
   \item La tâche du linguistique est d'éclairer ce que fait spontanément tout sujet parlant...
   \item ... et non de proscrire ou de promouvoir tel ou tel usage.
\end{itemize}

\subsection{De la difficulté d'étudier sa langue maternelle}
\begin{quote}
      La relation intime que chacun a nouée inconsciemment avec sa langue
maternelle est, dans une première étape, un obstacle à la recherche
raisonnée des principes de fonctionnement de celle-ci. La familiarité suscite
de l’intimidation. L’évidente compréhension que l’on a des énoncés peut,
dans un premier temps, détourner de poser des questions, empêcher de voir
que les données sont énigmatiques. Tout paraît aller de soi dans une langue
que l’on n’a pas apprise, mais que l’on sait.
\end{quote}
Delaveau, A. \& Kerleroux, F. (1985). Problèmes et exercices de syntaxe française. Paris : Colin. p. 5
%biblioooooooooo

\subsection{Objet de la linguistique}
\begin{itemize}
   \item L'objet de la linguistique : les usages des locuteurs,  les règles qui gouvernent ces usages.
   \item Une langue, par exemple le français, n'est pas observable, seuls les usages, les productions le sont et ce dans leur diversité, avec toutes leurs particularités individuelles.
   \item \textit{on n'étudie la langue qu'à travers son usage par ses locuteurs (on ne peut pas l'étudier en elle-même)}
\end{itemize}

\emph{Extrait du \textit{Cours de linguistique générale}}
\begin{itemize}
   \item Définition de la parole selon Saussure.
   \item les combinaisons par lesquelles le sujet parlant utilise le code de la langue.
   \item ("le système existe virtuellement dans chaque cerveau")
   \item en vue d'exprimer sa pensée personnelle.
\end{itemize}

\section{L'arbitraire du signe linguistique}

\textbf{Définition :}\\
   Terme utilisé par Seussure pour décrire la relation entre :
  \begin{itemize} 
      \item signifiant (forme sonore du signe)
      \item signifié (signification du signe)
  \end{itemize}
  %insérer fig2.png (illustration de la définition)
  %insérer fig3.png (exemple de signe linguistique)
\enquote{Le signifiant et le signifié sont indissociables comme le recto et le verso d'une feuille de parpier} Saussure.\\
Les deux sont indissociables car si l'on touche à la forme, on touche aussi au sens.\\
Lorsque signifiant et signifié sont séparés, on ne peut plus parler de signe linguistique.

\subsection{Notion d'arbitraire}
\textbf{Définition :}\\
   Toutes les langues se comportent comme des systèmes symboliques.
   Cela veut dire que les signes linguistiques se substituent aux choses.
   %insérer fig4.png (illustration de la définition)

\emph{Extrait du \textit{Cours de linguistique générale}}
L'idée de "Sœur" n'est liée par aucun rapport intérieur avec la suite de sons \/s-ö-r\/ qui lui sert de signifiant.
%revenir sur la notation phonétique
La diversité des langues montre qu'il n'y a pas de lien unique et nécessaire entre le son \/s-ö-r\/ et son signifié sœeur.
%revenir sur la notation phonétique
Ces liens découlent d'une convention partagée par tous les locuteurs de la même communauté linguistique pour garantir la stabilité des rapports.
L'arbitraire n'existe qu'entre le signifiant et le signifié.

\subsection{Les onomatopées, une exception?}
\textbf{Définition :}\\
      Mots dans lesquels la relation entre l'aspect phonique et le sens n'est pas arbitraire du fait que les sons imitent le sens.\\
      Souvent des interjections qui servent à offrir un équivalent du son perçu.\\
      \textit{Ex.: Dring, glouglou, gnangnan, pschitt, tic-tac, prout...}\\
      cris d'animaux : très bons condidatis à l'onomatopée.\\
      \textit{Ex.: coin-coin, meuh, hi-han, ...}

      Lien vers l'alphabet phonétique international : \textit{http://www.linguiste.org/}
      %possibilité d'éditer des hyperliens avec latex

\subsection{Cependant...}

Le coq italien fait \textit{kikiriki} et non \textit{cocorico}\\
Même les cris d'animaux n'ont pas les mêmes onomatopées selon la langue\\
Mais, si on regarde d'un peu plus près, on observe des constantes d'une langue à l'autre: \\
\begin{itemize}
         \item consonnes servant à reproduire le chant du coq ou du canard sont des consonnes vélaires : \/k\/
         \item consonne initiale du signe qui sert à reproduire le cri du chat : \/m\/ (\textit{il s'agit d'une constante dans pratiquement toutes les langues})
\end{itemize}

\subsection{Pourquoi ces veriations ?}
Les cris des animaux ne donnent pas lieu à des formes sonores identiques dans toutes les langues parce que : 
\begin{itemize}
         \item chaque langue a ses propres habitudes articulatoires.
         \item l'onomatopée n'est pas un double sonore pargait de ce qu'elle désigne, mais elle en bâtit uniquement les contours les plus caractéristiques.
\end{itemize}

\subsection{Conclusion}
\begin{itemize}
         \item Le lien entre la forme sonore des mots et ce qu'ils signifient est arbitraire.
         \item Chaque langue peut représeter n'import quelle signification par n'importe quelle combinaison de sons.
\end{itemize}

NB : les langues ne sont pas des nomenclatures : la réalité s'organise selon chaque langue. Chaque langue exerce ses propres contraintes.

La naissance d'un signe linguistique est la création simultanée d'un signifiant et d'un signifié, le signifié ne préexiste pas au signifiant ni l'inverse.

\subsection{De l'arbitraire à la motivation relative}

\textbf{Définitions :}\\
   \begin{itemize} 
      \item \emph{arbitraire} : un signe linguistique est dit non motivé, et donc arbitraire, quand sa forme, son Sa, n'évoque en rien son sens, son Se.
      \item \emph{motivation relative} : lien plus ou moins étroit entre un signe et la réalité qu'il désigne, entre la forme signifiante (Sa) et le signifié (Se).
   \end{itemize}

\emph{Extrait du \textit{Cours de linguistique générale}}
\begin{quote}
      Le principe fondamental de
l’arbitraire du signe n’empêche pas
de distinguer dans chaque langue ce
qui est radicalement arbitraire, c’est-
à-dire immotivé, de ce qui ne l’est
que relativement. Une partie
seulement des signes est absolument
arbitraire ; chez d’autres intervient
un phénomène qui permet de
reconnaître des degrés dans
l’arbitraire sans le supprimer : le
signe peut être relativement
motivé.
\end{quote}

\textbf{Illustration :}\\
Examinons ces deux séries :\\
\begin{tabular}{|l l}
   abricotier & abricot\\
   amandier & amande\\
   avocatier & avocat\\
   cerisier & cerise\\
   châtaignier & châtaigne\\
   citronnier & citron\\
   figuier & figue\\
   manguier & mangue\\
   marronnier & marron\\
   olivier & olive\\
   poirier & poire\\
   pommier & pomme\\
   prunier & prune\\
\end{tabular}
\\

On constate ici que le nom de l'arbre est construit sur la base du nom du fruit.\\
Le nom de l'arbre est alors dit \emph{partiellement motivé}.

\section{La non-fixité}
\subsection{Tendance de toutes les langues naturelles à évoluer}
Les effets de l'usage sur la langue.\\

\begin{itemize}
   \item Les signes linguistiques sont par nature livrés sans protection à l'usage des locuteurs. L'usage :
      \begin{itemize}
         \item les altères 
         \item les transforme
         \item en supprime 
         \item en produit de nouveaux
      \end{itemize}
   \item incessante activité de transformation des formes et d'établissement de noveaux rapports par oubli des anciens devenus non significatifs.
\end{itemize}

\emph{Extrait du \textit{Cours de linguistique générale}}
\begin{quote}
      Il n’y a pas d’exemple d’immobilité
absolue. Ce qui est absolu c’est le
principe du mouvement de la
langue dans le temps.
\end{quote}

\textbf{Illustrations :}
Entrées et sorties du Robert 2023 :\\
%insérer fig5.png

 \subsection{Pas de fixité, mais un principe d'évolution}
\begin{itemize} 
         \item La langue évolue
         \item La néologie est un processus immanent
\end{itemize}

\textbf{Une idée reçue !}
\begin{itemize}
         \item Selon certains, l'évolution serait une dégénérescence et non un progrès, une forme de "corruption" de la langue.
         \item Or, une langue doit évoluer pour ne pas mourir
\end{itemize}

\textbf{Illustration :}\\
Le latin n’est pas mort parce que
tous ses locuteurs seraient morts.
Au contraire, il a continué à
évoluer sur son propre territoire
et sur les territoires conquis par
les romains et il a donné
naissance par fragmentation
dialectale à ce qu’on appelle les
langues romanes (français, italien,
espagnol, portugais, roumain,
etc.)

\subsection{Le lexique évolue vite, mais la syntaxe change aussi}
\begin{itemize}
         \item On dit souvent que c'est le lexique, la partie d'une langue, qui évolue le plus rapidement.
         \item La néologie en est la marque la plus spectaculaire car la plus visible. 
         \item Mais la syntaxe évolue elle aussi.
\end{itemize}

\textbf{Exemples d'évolutions syntaxiques du français oral :}
\begin{itemize}
         \item Forme interrogative marquée par l'intonation : \\
            \enquote{Léa nous rejoint pour manger?}
         \item Disparition du subjonctif imparfait.
\end{itemize}

\subsection{Conclusion}
Contrairement à ce que souhaireraient certains pour lesquels l'idéal serait la fixité de la langue, toutes les langues subissent l'influence d'autres langues en contact avec elles.







