% cours.tex s1.tex s2.tex makefile

\section{Extraites à lire}
\subsection{Gary... ;;;à compléter;;;}
\subsection{F. de Saussure (1972), à propos des différences entre langage et langue}
\subsubsection{Langue}
\begin{itemize}
   \item produit social de la faculté de langage
   \item ensemble de conventions adoptées par le corps social
   \item acquis
   \item ;;; \textit{jusqu'a 17 mois, le bébé est capable de discriminer tous les phonèmes que l'humain peut produire}
\end{itemize}
\subsubsection{Langage}
\begin{itemize}
   \item Faculté cognitive innée propre à l'homme
\end{itemize}

\part{Chapitre 1}
\subsection{Objectifs du cours}
\begin{itemize}
   \item Cours de linguistique générale
   \item ;;;
   \item L'arbitraire du signe linguistique
\end{itemize}

\section{Cours de linguistique générale}
Rédigé à partir des notes d'étudiants de Saussure. (Rédigé par Bally et Sechehaye à partir de leurs notes et celles de leurs camarades)

\subsection{Une œuvre bien particulière}
il a proposé le contenu mais pas rédigé

\subsection{Une publication incontournable}
Première fois que la recherche sur le langage et les langues tente de :
\begin{itemize}
   \item penser rigoureusement les propriétés de son objet
   \item fixer les limites de son champ
\end{itemize}

\subsection{objectifs du cours}
\begin{itemize}
   \item pas étudier l'ouvrage
   \item mais de faire apparaître quelques une des propriétés qui y sont bien décrites :;;;;
\end{itemize}

\section{Une distinction importante : en mention ou en usage}

Un mot est dit "en usage" lorsqu'il est utilisé pour revoyer à un objet extérieur à lui-même (dans un cas on parle du monde).
\textit{Exemple : je ne mange \textbf{jamais} de blé}
Les mots en usage sont les mots \textbf{avec lesquels} on parle.

Un mot est dit "en mention" lorsqu'il est utilisé pour ne renvoyer qu'à lui-même (dans un cas on parle de mots).
\textit{Exemple: \textbf{"jamais"} est un adverbe.}
Il s'agit alors d'un usage métalinguistique.
Les mots en mention sont les mots \textbf{dont} on parle.
Pour les examens, on souligne ou met entre guillemet les mots en mention.

\subsection{une illustration de l'opposition en usage / en mention}
bling.hy;;;

\section{Une langue : l'objet intime et familier à tout locuteur}
\begin{itemize}
   \item Tout locuteur a une expérience quotidienne de sa langue : il sait parler.
   \item Mais il ignore l'histoire de sa langue et les règles qu'il applique pour parler.
   \item Cependant tout locuteur applique les règles de sa langue en permanence.
\end{itemize}

\subsection{Extrait du \textit{Cours de linguistique générale}}
\begin{itemize}
   \item "Ceux-là mêmes qui en font un usage journalier [de la langue] l'ignorent profondément"
   \item La tâche du linguistique est d'éclairer ce que fait spontanément tout sujet parlant...
   \item ... et non de proscrire ou de promouvoir tel ou tel usage.
\end{itemize}

\subsection{De la difficulté d'étudier sa langue maternelle}
;;; extrait de 

Delaveau, A. \& Kerleroux, F. (1985). Problèmes et exercices de syntaxe française. Paris : Colin. p. 5 ;;;penser à justifier à droite pour les références bibliographiques;;;

\subsection{Objet de la linguistique}
\begin{itemize}
   \item L'objet de la linguistique : les usages des locuteurs,  les règles qui gouvernent ces usages.
   \item Une langue, par exemple le français, n'est pas observable, seuls les usages, les productions le sont et ce dans leur diversité, avec toutes leurs particularités individuelles.
   \item \textit{on n'étudie la langue qu'à travers son usage par ses locuteurs (on ne peut pas l'étudier en elle-même)}
\end{itemize}

\subsubsection{Extrait du \textit{Cours de linguistique générale}}
\begin{itemize}
   \item Définition de la parole selon Saussure.
   \item les combinaisons par lesquelles le sujet parlant utilise le code de la langue.
   \item ("le système existe virtuellement dans chaque cerveau")
   \item en vue d'exprimer sa pensée personnelle.
\end{itemize}

\section{L'arbitraire du signe linguistique}
\begin{def}
   Terme utilisé par Seussure pour décrire la relation entre :
  \begin{itemize} 
      \item signifiant (forme sonore du signe)
      \item signifié (signification du signe)
  \end{itemize}
  ;;;récupérer les figures sur le cours sur iris;;;
\end{def}
"Le signifiant et le signifié sont indissociables comme le recto et le verso d'une feuille de parpier" Saussure.\\
Les deux sont indissociables car si l'on touche à la forme, on touche aussi au sens.\\
Lorsque signifiant et signifié sont séparés, on ne peut pas parler de signe linguistique.

\subsection{Notion d'arbitraire}
\begin{def}
   Toutes les langues se comportent comme des systèmes symboliques.
   Cela veut dire que les signes linguistiques se substituent aux choses.
   ;;; récupérer les figures sur iris;;;
\end{def}

\subsection{Extrait du \textit{Cours de linguistique générale}}
L'idée de "Sœur" n'est liée par aucun rapport intérieur avec la suite de sons \/s-ö-r\/ qui lui sert de signifiant.

La diversité des langues montre qu'il n'y a pas de lien unique et nécessaire entre le son \/s-ö-r\/ et son signifié sœeur.
Ces liens découlent d'une convention partagée par tous les locuteurs de la même communauté linguistique pour garantir la stabilité des rapports.
L'arbitraire n'existe qu'entre le signifiant et le signifié.

\subsection{Les onomatopées, une exception?}
\begin{def}
      Mots dans lesquels la relation entre l'aspect phonique et le sens n'est pas arbitraire du fait que les sons imitent le sens.\\
      Souvent des interjections qui servent à offrir un équivalent du son perçu.\\
      \textit{Ex.: Dring, glouglou, gnangnan, pschitt, tic-tac, prout...}\\
      cris d'animaux : très bons condidatis à l'onomatopée.\\
      \textit{Ex.: coin-coin, meuh, hi-han, ...}
      ;;;figure sur iris;;;
      ;;;lien vers l'alphabet phonétique internationnal sur iris;;;
\end{def}

\subsection{Cependant...}

Le coq italien fait \textit{kikiriki} et non \textit{cocorico}
;;;figure;;;
Même les cris d'animaux n'ont pas les mêmes onomatopées selon la langue\\
Mais, si on regarde d'un peu plus près, on observe des constantes d'une langue à l'autre: \\
\begin{itemize}
         \item consonnes servant à reproduire le chant du coq ou du canard sont;;;
\end{itemize}

\subsection{Pourquoi ces veriations ?}
Les cris des animaux ne donnent pas lieu à des formes sonores identiques dans toutes les langues parce que : 
\begin{itemize}
         \item chaque langue a ses propres habitudes articulatoires.
         \item l'onomatopée n'est pas un double sonore pargait de ce qu'elle désigne, mais elle bâtit ;;;
\end{itemize}

\section{Conclusion}
\begin{itemize}
         \item Le lien entre la forme sonore des mots et ce qu'ils signifient est arbitraire.
         \item Chaque langue peut représeter n'import quelle signification par n'importe quelle combinaison de sons.
\end{itemize}

NB : les langues ne sont pas des nomenclatures : la réalité s'organise selon chaque langue. Chaque langue exerce ses propres contraintes.

La naissance d'un signe linguistique est la création simultanée d'un signifiant et d'un signifié, le signifié ne préexiste pas au signifiant ni l'inverse.

\subsection{De l'arbitraire à la motivation relative}

\begin{def}
   \begin{itemize} 
            \item arbitraire : un signe linguistique est dit non motivé, et donc arbitraire, quand sa forme, son Sa, n'évoque rien;;;
   \end{itemize}
\end{def}

;;;illustration de la motivation sur iris;;;

\section{La non fixité}
\subsection{Tendance de toutes les langues naturelles à évoluer}
Les effets de l'usage sur la langue.\\

Les signes linguistiques sont par nature livrés sans protection à l'usage des locuteurs. L'usage :
\begin{itemize}
         \item les altères 
         \item les transforme
         \item en supprime 
         \item en produit de nouveaux
\end{itemize}

incessante activité de transformation des formes et d'établissement de noveaux rapports par oubli des anciens devenus non significatifs.


\subsection{Extrait du \textit{Cours de linguistique générale}}
Il n'y a pas d'exemple d'immobilité absolue. Ce qui est absolu c'est le principe du mouvment;;;
;;;illustration sur iris;;;
 \subsection{Pas de fixité, mais un principe d'évolution}
\begin{itemize} 
         \item La langue évolue
         \item La néologie est un processus immanent
\end{itemize}

Une idée reçue !
\begin{itemize}
         \item Selon certains, l'évolution serait une dégénérescence et non un progrès, une forme de "corruption" de la langue.
         \item Or, une langue doit évoluer pour ne pas mourir
\end{itemize}









