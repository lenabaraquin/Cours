\part{Chapitre 6 - l'orthographe du français}
\section*{Introduction à la réflexion sur l'orthographe}
sketch pour montrer que pour une forme orale, il existe plein de possibilité de la retranscrire par une forme graphique
\section{Définition}
orthographe vient du grec /orthos/ droit et /graphein/ écrire
Des questions sur "Quelle est la \textbf{bonne} façon d'écrire?"
   régie par des lois dont les dictionnaires (lexical) et les grammaires (grammatical) rendent compte
   tout ce qui s'en écarte est stigmatisé (erreur, faute)

La langue ne dépend pas de l'orthographe : des langues peuvent exister sans orthographe
Une orthographe est une norme, convention destinée à retranscrire l'oral à l'écrit
Une orthographe est relative à un contexte socio-historique
\subsection{nénufar ou nénuphar?}
De persan \textbf{nilufar} qui est devenu nénuphar en 1935 (complexification de l'orthographe)
Petit historique de l'orthographe de nénuf-phar:
\begin{itemize}
   \item Proust : nénufar
   \item Genevoix : nénufar
   \item Claudel : nénuphar
\end{itemize}

\subsection{Une bonne orthographe?}
une graphie correcte est une graphie qui se rapproche à la norme, 
    or, si la norme est l'oral, le français écrit s'éloigne beaucoup de l'oral

Blanche-Benveniste:
\XXX{[les difficultés de l'orthographe du français] s'explique par ces manques et par la complexité des divers procédés auxquels la langue a dû recourir pour compler ce déficit initial.}

Pas assez de lettre latine pour retranscrire les sons du français:
\begin{itemize}
   \item création de nouvelles lettres
   \item ajour de signes diacritiques sur des lettres existantes
   \item regroupement de lettres existantes
   \item multiplication des groupements de lettre pour retranscrire un même son
\end{itemize}

\begin{def}
   Diacritique : qui sert à distinguer
\end{def}
signe diacritique : accents, cédille, ...
\section{Complexité de l'orthographe}
\subsection{L'Académie Française}
deux manières d'écrire les mots:
\begin{itemize}
   \item tenir compte de leur origine
   \item tenter de les transcrire phonétiquement
\end{itemize}

c'est tellement des vieux crouteux que dans leur première édition : l'académie doit préférer \XXX{l'ancienne orthographe, qui distingue les gens de lettrees d'avec les ignorants et les simples femmes}.

Plusieurs réformes de l'orthographe depuis le XVIème siècle:
\begin{itemize}
   \item 1530: écrire comme on prononce
   \item 1740: remplacement de y par i
   \item 1835: suppression des lettre grecques th, ph dans certains mots
   \item 1963: suppression des doubles consonnes de certains mots (mais cette réforme n'a pas abouti)
   \item 1990: Dernière réforme en date
\end{itemize}
\section{Dernière réforme en date : la réforme de 1990}
quelques exemples :
\begin{itemize}
   \item allègement ou allégement
   \item allègrement ou allégrement
   \item aigüe ou aiguë
   \item coût ou cout
   \item entraîner ou entrainer
   \item les empruns forment leurs pluriels comme les mots français:
   \begin{itemize} 
      \item maximums et pas maxima
      \item barmans et pas barmen
   \end{itemize}
\end{itemize}
Tolérance dans l'acceptation des formes non standard à l'orthographe 
\section{Des freins à l'usage de l'orthographe simplifiée}
L'application des réformes de simplification se heurte à des rétissance du publique qui reste attaché aux normes de l'ancienne orthographe
plusieurs exemples de critiques :
\begin{itemize}
   \item la qualité \textit{esthétique} de la langue (toujours aussi peu de sens)
   \item peur de perte de la marque est étymologique (ce que l'on pense être un marquage éthimologique n'en est pas toujours un : "nénuphar", "poids", "fantaisie"...)
\end{itemize}

