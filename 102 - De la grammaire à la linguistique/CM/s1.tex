% cours.tex s1.tex s2.tex makefile

\part{Cours 0 - Introduction}
\section{Généralités}
\subsection{Le mystère du langage humain}
\begin{itemize}
   \item juste un bruit mais qui a du sens
   \item apparition du langage : sous quelles conditions et à quel moment?
   \item facilité d'acquisition du langage dans la petite enfance
   \item cognition
   \item la capacité de la parole se réalise de manière différente selon les communautés linguistiques (contrairement à la vue, l'ouïe...)
   \item similitudes et différences entre fonctionnnement des langues (grammaire)
   \item toute une culture avec une trentaine de phonèmes (en français) (mais d'autres n'en ont qu'une quinzaine) \textit{combien de phonèmes au minimum pour une langue fonctionnelle?}
\end{itemize}

\subsection{Objectifs du cours}
L'objectif de ce cours est de situer la linguistique par rapport à la grammaire.
En effet, la linguistique est une nouvelle discipline, un nouvel objet d'étude, en revanche, c'est un objet d'étude qui paraît familier du fait de notre usage quotidien d'une (ou de plusieurs) langue.

\subsection{Programme}
\begin{enumerate}
   \item deux propriétés fondamentales des langues
   \item les unités d'une langue : le cas du francais
   \item évolution de l'emploi des prépositions
   \item la phrase
   \item le complément circonstanciel en discussion
   \item l'orthographe du français
   \item l'accord du participe passé 
   \item la néologie \textit{s'intéresser à la veille néologique (tanter de trouver des bases de données)}
   \item le français parlé
   \item langue et variations
   \item Conclusion
\end{enumerate}

\subsection{MCC}
en 2 parties:
\begin{enumerate}
   \item 5 tests tout au long du semestre
      \begin{itemize} 
         \item tests de 1 à 4 pour 10\% (note de la première tantative) (dispo pendant 2 semaines)
            \begin{enumerate} 
               \item identification des classes syntaxiques (semaine 3)-ch 2
               \item identification des fonctions des constituants (semaine 6)-ch 5
               \item accord du participe passé (semaine 8)-ch 7
               \item questions de cours portant sur l'ensembles des chapitres (semaine 10)
            \end{enumerate}
         \item test 5 (avec temps imparti) pour 40\% (dispo pendant seulement 1 semaine (en 30min)) 15 décembre
      \end{itemize}
   \item test en janvier pour 50\%
\end{enumerate}

\section{De la grammaire à la linguistiques}
\subsection{Deux disciplines qui n'ont pas les mêmes objectives}
\begin{itemize}
   \item la grammaire s'intéresse à la pédagogie et à la langue correcte (usage normatif)
   \item la linguistique a pour objectif de comprendre le fonctionnement de la langue (usage descriptif)
\end{itemize}
\subsection{Pourquoi partir de la grammaire?}
\begin{itemize}
   \item   la grammaire en tant que discipline est familiaire au locuteurs parce que les livres de grammaires accompagnent les élèves pendant toute leur scolarité
   \item   les livres de grammaire existent depuis des sciècles 
   \item   les livres de grammaire sont les premières manifestations d'une réflexion sur la langue qui n'a pas attendu la naissance de la discipline linguistique pour s'exercer (les livres permettent de conserver les acquis)
   \item   les livres de grammaire fournissent un ensemble d'informatinos constituant un point de départ solide pour toute réflexion sur la langue
\end{itemize}
\subsection{Conclusion}
   A la différence de la grammaire, la linguistique est une discipline récente (XIXème sciècle)

\section{Qu'est-ce que la linguistique}
\begin{itemize}
   \item la linguistique est une discipline qui étudie les langues et notamment les différents usages d'une langue donnée, cest usages pouvant parfois s'éloignenet considérablement des règles réunies dans la grammaire de cette langue
      \begin{itemize} 
         \item l'usage de la langue ne suit pas toujours la grammaire
         \item par exemple la formation de la négation 
         \item ou alors "au coiffeur" et pas "chez le coiffeur"
         \item "malaise voyageur" au lieu de "malaise d'un voyageur" (association de deux nom sans élément de relation (propositions))
         \item   \textit{est-ce un usage incorrect et donc négligeable ou alors est-ce une évolution de la langue}
      \end{itemize}
   \item la linguistique s'interesse à la description de toutes les langues sans exceptions, sans préjugés culturels, car tout système de communication linguistique mérite d'être décrit, même en l'absence d'une tradition écrite, même les dialectes.
\end{itemize}
\section{Définition de la linguistique}
\begin{itemize}
   \item la linguistique est une science récente
   \item la linguistique s'est développée à la fin du XIXeme sciècle
   \item la linguistique se donne comme objet la connaissance du langage à travers l'étude des langues
\end{itemize}
la linguistique est donc l'étude scientifique du langage à travers la description des langues naturelles

\subsection{Langage et langues}
définitions : 
\begin{itemize}
   \item  langage
      \begin{itemize} 
         \item faculté humaine : faculté de parler. C'est l'aptitude des humains à communiquer au moyen de signes linguistiques.
         \item caractéristique \textbf{universelle} et \textbf{essentielle} de l'homme : c'est ce qui caractérise et distingue l'humain des autres êtres vivants, notamment des autres espèces animales
      \end{itemize}
   \item langues
      \begin{itemize} 
         \item instruments de communication extrêmement performant. Ce sont des systèmes de signes et de règlesqui sont spécifiques aux membres d'une communauté et qui leur permettent de communiquer entre eux.
            \begin{itemize} 
               \item un signe peut être un mot (mais aussi des phonèmes ou des morphèmes)
               \item pour constituer une langue, il est nécessaire que les unités linguistique sont agencées avec des règles - on a alors un système linguistique (ensemble cohérent).
            \end{itemize}
         \item la langue peut se décrire par elle même 
         \item la faculté humaine de langage se réalise dans chaque communauté d'une manière particulière d'où l'existence de différentes langues : l'anglais, le français, le basque, le swahili... (il y a entre 6~000 et 7~000 langues parlées actuellement dans le monde)
      \end{itemize}
   \item parole 
      \begin{itemize} 
         \item  utilisation concrète d'un système linguistique par un locuteur donné dans une situation de communication donnée.
         \item  les usages relèvent de la parole
      \end{itemize}
\end{itemize}

\section{Qu'observe le linguiste? Les usages d'une langue donnée}
\begin{itemize}
   \item  une langue n'est pas observable directement, seuls les usages de cette langue peuvent être observés.
   \item  les usages sont les productions concrètes des locuteurs qui utilisent une langue donnée, à l'oral et à l'écrit, dans toutes sortes de contextes:
      \begin{itemize} 
         \item conversations entre amis, cours émissions de radio, discours politiques... (contrairement à la grammaire pendant plusieurs sciècles qui s'interressait seulement aux grands auteurs)
         \item articles de presse, copies d'étudiants, documentation d'entreprise, romans, essais, SMS...
      \end{itemize}
   \item  c'est à travers les usages qu'on étudie les langues
   \item \textit{pourquoi le système tolère certains usages impropres et pas d'autres: "j'ai pas faim", "je ne faim pas"}

   \item l'objet premier de la linguistique est donc constitué par les usages des locuteurs 
   \item un des objectifs de la linguistique est de découvrir les règles qui gouvernent ces usages
   \item sachant que les productions des locuteurs varient beaucoup :
      \begin{itemize} 
         \item selon la situation sociale dans laquelle le locuteur s'exprime 
         \item selon que le locuteur urilise la langue à l'oral ou à l'écrit
         \item mais aussi selon la région où il se trouve
      \end{itemize}
   \item  la conformité aux règles est nécessaire pour la compréhension
\end{itemize}
\section{Objectif plus spécifique du cours}
Décrir deux des propriétés fondamentales de toutes les langues du monde:
\begin{enumerate}
   \item l'arbitraire du signe : le lien entre la forme sonore des mots et ce qu'ils signifient est complètement arbitraire, chaque langue représente n'importe quelle signication par n'importe quelle combinaison de sons
   \item la non fixité : il existe une incessante activité de transformation des formes, d'établissmeent de nouveaux rapports par oubli des anciens devenus non significatifs.
\end{enumerate}


\section{Bibliographie}

\textbf{Une grammaire :}
\begin{itemize}
   \item Grevisse, M. \& Goosse, A. (2016). Le Bon Usage - Grammaire française. Bruxelles : DeBoeck, Louvain-la Neuve : Duclot. 16\up{ème} édition.
   \item Riegel, M., Pelat, J.C. \& Rioul, F. (2018). Grammaire méthodique du français. Paris : Presses Universitaires de France. 7\up{ème} édition.
\end{itemize}

\textbf{Un dictionnaire :}
\begin{itemize}
   \item Dictionnaire \textit{Le Petit Robert de la langue française} (version numérique accessible depuis l'ENT).
\end{itemize}

\textbf{À lire pour la semaine prochaine :}
\begin{itemize}
   \item Gary-Prieur, M.-N. (1985). De la grammaire à la linguistique. L'étude de la phrase. Paris : A. Colin.
   \item Ferdinand de Saussure (1972) [1916]. Cours de linguistique générale. Paris : Payot.
\end{itemize}
