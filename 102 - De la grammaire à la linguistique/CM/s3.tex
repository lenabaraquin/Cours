\part{Capitre 2 - Les unités d'une langue - le cas du français}

\section*{plan du cours}
\begin{enumerate}
   \item les unités de la grammaire traditionnelle
   \item deux études de cas
   \item les unités de la linguistique
   \item les classes syntaxiques du français
\end{enumerate}

\section{Les unités de la grammaire}
\subsection{Objectif d'une grammaire : classer les unités d'une langue}

Depuis ses origines et jusqu'à une époque récente, la grammaire a pour tâche essentielle le classement des mots d'une langue dans un petit nombre de catégories pertinentes pour la description de cette langue : les parties du discours.
\begin{itemize}
   \item feuilles, rivière ... sont des noms
   ... suite sur iris
\end{itemize}

Le terme "partie du discours" appartient à la grammaire traditionnelle. ... suite sur iris

;;;
Pourquoi des catégories? 
   pour formuler des règles très générales (abstraite), qui ont alors une capacité d'explication forte
;;;

\textbf{Dans les grammaires...}
\begin{itemize}
   \item un nom est un mot qui désigne une personne, une chose ou un lieu
   \item un adjectif exprime une qualité 
   \item un verbe exprime une action ou un état
\end{itemize}

;;;
Les classes grammaticales dépendent de la sémentique selon les grammaires (déscpription sémentique)
   ;;;

\textbf{Et pourtant...}
\begin{itemize}
   \item gentillesse est un nom qui exprime une qualité
   ....

\end{itemize}
 
;;;
Une approche purement sémentique n'est donc pas le plus adapté
;;;

\subsection{Notation}
L'astérisque *.
Placé avant un mot ou une phrase, il indique que le mot ou la phrase ne sont pas acceptables, ne font pas partie des mots ou des phrases ....

\textbf{Dans les grammaires :}
\begin{quote}
   Les articles ont ....
\end{quote}

Le déterminant est défini comme un élément servant à donner le genre et le nombre au nom.
(pourtant le genre est intrisèque au nom et le nombre dépend de la situation).

\textbf{D'autres critères :}
\begin{itemize}
   ....
\end{itemize}

La fonction du déterminant est d'actualiser le nom.
On dit qu'un mot est actualisé lorsqu'il est utilisé dans le discours.
Par opposition à un mot virtuel.

\begin{itemize}
   \item \textbf{Actuel :} c'est le mot réalisé dans la parole, à travers un emploi particulier.
   \item \textbf{Virtuel :} c'est le mot considéré en langue.
\end{itemize}

\textbf{De virtuel à actuel}
\begin{itemize}
   \item Le dictionnaire contient les mots virtuels, rangés comme des vêtements dans une armoire.
   \item On actualise ces mots dès lors qu'on les utilise dans une phrase.
\end{itemize}

\section{études de cas}
\textbf{Exemples :}
\begin{itemize}
   \item Le tigre est un animal menacé.
   \item L'éleveur craint de ne pas avoir assez de maïs pour nourrir sa vache.
\end{itemize}

\textbf{Conclusion :}
employés dans un énoncé, \textit{tigre} ou \textit{vache} exigent un déterminant, choisi en fonction du nom, masculin pour \textit{tigre}, féminin pour \textit{vache}.

\textbf{Dans les grammaires :}
\begin{itemize}
   \item adjectifs possessifs : mon, ton, son, sa, tes ...
   \item adfectifs démonstratifs : ce, cette, ces...
   \item adjectifs qualificatifs : gentil, rouge...
\end{itemize}

\textbf{Et pourtant}
\begin{itemize}
   \item les déterminants servent à actualiser le nom : ils sont obligatoires et suffisants pour l'emploi d'un nom en position de sujet:\\
      \textit{Ex.: Le livre est sur la table.}
   \item les adjectifs modifient le nom : ils lui apportent certaines caractéristiques de façon toujours facultative.\\
      \textit{Ex.: *Livre rouge est sur la table}
\end{itemize}
L'adjectif n'a pas la capacité d'actualiser le nom.
Par conséquent : \textit{*Livre rouge est sur la table}

;;;
On va donc arrêter de ranger les adjectifs démonstratifs (qui permettent d'actualiser le nom) dans la même catégorie que les adjectifs qualificatifs qui ne peuvent pas avoir cette fonction.
;;;

\textbf{Au sens étymologique...}
le mot pronom signifie "remplace le nom".

\textbf{Mais cette étymologie du mot "pronom est trompeuse":}\\
      Le pronom ne remplace pas le nom tout seul, mais le nom accompagné de son déterminant, le nom actualisé par le déterminant.
      \textit{Exemples : 
     \begin{enumerate} 
         \item Ses collègues pensent qu'il va démissionner.
         \item Ils pensent qu'il va démissionner.
         \item Il pense à sa démission depuis des mois.
         \item Il y pense depuis des mois.
     \end{enumerate}}\\

     Certains pronoms ne remplacent rien :
     \textit{Ex.: je, personne, rien,...}
....



Les définitions des grammaires traditionnelles sont donc lacunaires et réductrices car elles ne couvrent pas tous les faits de langues qu'elles sont sensées décrire.

\subsection{Une distinction importante}
Il existe 2 types de pronoms personnels :
\begin{itemize}
   \item Les pronoms personnels conjoints ou clitiques.
   \item les pronoms personnels disjoints ou non clitiques.
\end{itemize}

.... insérer figure sur iris
P1 : locuteur
P2 : iterlocuteur
P3 : celui/ceux dont on parle

la variation du pronom selon le cas est un vestige des déclinaisons latines
;;;; fouiller la "fonction syntaxique" du pronom

opposition conjoint/disjoint
\begin{itemize}
   \item Conjoints ou clitiques : plus grande variabilité formelle que les disjoints
   \item Le disjoint ou non clitique (autonome) peut être éloigné du verve, le clitique non 
   ;;;;
  \begin{itemize} 
   \item formes clitiques dépendantes du verbe: s'appuient sur le verbe, entre autre, à cause de leur manque d'autonomie accentuelle.
   \item sont des formes inaccentuées - elles ne ;;;;
  \end{itemize}
\end{itemize}

;;;;;;;;;;
 

\section{Les unités de la linguistique}
\begin{itemize}
   \item 1er critère opératoire : le critère morphologique.
   \item 2eme citère opératoire : le citère syntaxique ou distributionnel.
   \item Application des critères pour disctiguer les adjectifs des adverbes.
\end{itemize}

on s'interesse aux indices formels qui permettent ....
\subsection{critère morphologique}
;;;;
\subsection{critère syntaxique}
Les mots n'occupent pas les mêmes places dans la phrase selon leur catégorie.\\
       En français, le déterminant est avant le nom et jamais ailleurs.
       ce trait;;;;
       ;;;;;;

       lorsque l'on emet un enoncé, on combine un certain nombre de termes (axe synthagmatique)



       contexte (en syntaxe) d'une unité dans une suite, ce qu'il reste d'un point de vue syntaxique lorsue l'on retire l'unité étudiée du cette suite

     ;;;;  paradigme : classe d'équivalence;;;;







Exemple :\textit{Je m'occupe des enfants ce soir}
Pour remplasser \emph{des}:
\begin{itemize}
   \item des
   \item *les 
   \item *des vous
   \item de nos
\end{itemize}

\subsection{Application des critères pour distiguer les adjectifs des adverbes}
\textbf{Critères mortphologique}
\begin{itemize}
;;;;récup fig. sur iris
\end{itemize}

Les adverbes peuvent se retrouver n'importe où dans la phrase (pas tous mais la majorité)


Le participe passé peut agir comme un adjctif ou comme un verbe, mais dans tous les cas il s'accorde en genre et en nombre au nom qu'il qualifie



Le participe présent peut être coordoné à un adjectif s'il s'agit d'un emploi adjectival


tout est un terme extrêmement complexe car il peut etre utiliseé comme nom, ..., \textbf{adverbe}
règle d'accord de l'adverbe \textit{tout} :
Devant un adjectif féminin à initiale consonantique en raison de l'euphonie (facilité de prononciation)



Les classes lexicales renvoient à des catégories conceptuelles.
les classes grammaticales renvoient vers des catégories abstraites
