\cite{Pep20} \\
      Cet article scientifique d'Erwan Pépiot et d'Aron Arnold porte sur les différence de prosodie dans un corpus de locuteurices bilingues anglais - français en fonction de leur genre. Les auteurs font l'hypothèse que les paramètre prosodiques qui permettent de genrer un individu ne dépendent pas seulement de données anatomiques et physiologique mais aussi de données culturelles. Pour montrer cette hypothèse,??? les auteurs font appel à un corpus de XXX bilingues XXX pour montrer que 
      Erwan Pépiot est chercheur en XXX à l'université Paris 8, a fait sa thèse sur...XXX et Aron Arnold est chercheur en XXX à l'université catholique de Louvain, il a fait sa thèse sur...XXX.
      XXX lien avec la problématique XXX


      (plutot traite de ...)montre l'implication des normes culturelles de genre dans les différences des paramètres prosodiques de la voix des locuteurs en fonction de leur genre au travers d'une étude sur un corpus de locuteurices bilingues anglais - français.
      ;présentation des auteurs;
      Il permet de montrer que (à la limite faire un lien avec la problématique)\\

\cite{Bro18} \\
      Cet article de presse d'Émilie Brouze traite de 
      ;présentation de l'autrice;
      Il s'intègre dans la problématique car...\\

\cite{podcast} \\
      Ce podcast traite de 
      ;présentation de ...;
      Il s'intègre dans la problématique car...\\

%Il faut absolument un chapitre d'ouvrage collectif!!!!!!!

\cite{Boe75} \\
      Cet article scientifique de Louis-Jean Boë, Michel Contini et Hippolyte Rakotofiringa traite de 
      présentation des auteurs
      Il s'intègre dans la problématique car...\\

\cite{Lek16} ????

\cite{Pep16}

\cite{}
      
\printbibliography
