\cite{Pep20} \\
      Cet article scientifique d'Erwan Pépiot et d'Aron Arnold porte sur les différences de prosodie dans un corpus de locuteurices bilingues anglais - français en fonction de leur genre.\\
      Erwan Pépiot est chercheur en linguistique à l'université Paris 8 et Aron Arnold est chercheur en études de genre et en sociophonétique à l'université catholique de Louvain \\

\cite{Bro18} \\
      Cet article de presse d'Émilie Brouze est une interview de Aron Arnold (déja présenté dans \cite{Pep20}) à propos de l'implication du genre dans la voix.\\
      Émilie Brouze est autrice et journaliste à l'Obs.

\cite{podcast} \\
      Ce podcast traite de 
      ;présentation de ...;
      Il s'intègre dans la problématique car...\\

\cite{Boe75} \\
      Cet article scientifique de Louis-Jean Boë, Michel Contini et Hippolyte Rakotofiringa traite de XXX\\
      Louis-Jean Boë, Michel Contini et Hippolyte Rakotofiringa sont tous trois chercheur en sciences de la parole à l'université de Grenoble Alpes.\\

\cite{Pep16} \\
      Dans cette conférence, Erwan Pépiot montre, en se basant sur sa thèse, que ...\\
      Erwan Pépiot est présenté dans la description de \cite{Pep20}.\\

\cite{Can15} \\
      Cet article de Maria Candea et Cyrill Trimaille est, comme son nom l'indique, une introduction à la sociophonétique, une discipline encore très jeune (peu de littérature y est donc rattaché pour le moment) qui traite de l'impact social de la phonétique.\\
      ; présentation ;
      Cet article ne s'intègre pas en tant que tel dans le sujet que j'ai choisi, il m'a en revanche été utile pour avoir un aperçu de l'état de l'art de cette discipline (dans laquelle s'integre mon sujet) ainsi que pour trouver l'article \cite{Pep20}.\\
      
\cite{DiC13}\\
      Ce chapitre d'ouvrage traite de la \say{matérialité acoustique et auditive de la prosodie}, il permet d'expliquer, sous l'angle de la physique ;;;accoustique;;;, différents paramètres prosodiques comme la fréquence fondammentale, les formants de voyelles, le timbre.\\
      ;;; présentation ;;;
      Ce chapitre d'ouvrage, au même titre que l'article \cite{Can15}, ne s'intègre pas directement dans le sujet que j'ai choisi mais il m'a aidé à comprendre des notions abordées dans d'autres articles.\\

\cite{Gar22}\\
      Ce mémoire traite de \\
      Mona Garczarek est \\

\printbibliography
