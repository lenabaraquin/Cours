\parencite{Pep20} \\
      Cet article scientifique d'Erwan Pépiot et d'Aron Arnold porte sur les différences de prosodie dans un corpus de locuteurices bilingues anglais - français en fonction de leur genre.\\
      Erwan Pépiot est chercheur en linguistique à l'université Paris 8 et Aron Arnold est chercheur en études de genre et en sociophonétique à l'université catholique de Louvain.\\

\parencite{Bro18} \\
      Cet article de presse d'Émilie Brouze est une interview de Aron Arnold (déjà présenté dans \parencite{Pep20}) à propos de l'implication du genre dans la voix.\\
      Émilie Brouze est autrice et journaliste à l'Obs.\\

%\parencite{podcast} \\
%      Ce podcast traite du genre dans la voix chantée. Il fait intervenir deux chanteurs ainsi qu'un enseignant de littérature pour XXX; respectivement Gérard Lesne, Artur H et Jérôme Solal.
%      Ce podcast est produit par Jérôme Sandlarz, producteur chez radio france et réalisé par Somany Na, Vincent Abouchar, Olivier Bétard et Hervé Marchon, réalisateurices chez radio france.

\parencite{video} \\
      Cette vidéo d'Amélia Huff est un guide didactique de féminisation de la voix. Amélia Huff y explique que la modification de la fréquence du larynx, à fréquence de vibration des cordes vocales similaire, a un fort impact sur la perception du genre vocal.\\
      Amelia Huff, publiquement connue sous le nom de Zhea Erose, est une vidéaste et musicienne. Elle est à l'origine de la chaine youtube TransVoiceLessons.\\

\parencite{Boe75} \\
      Cet article scientifique de Louis-Jean Boë, Michel Contini et Hippolyte Rakotofiringa est une étude de différents paramètres prosodiques (fréquence laryngienne, notée $F_1$, et durée des sons voisés) sur un corpus de 30 hommes et 30 femmes francophones.\\
      Louis-Jean Boë, Michel Contini et Hippolyte Rakotofiringa sont tous trois chercheurs en sciences de la parole à l'université de Grenoble Alpes.\\

\parencite{Pep16} \\
      Dans cette acte de conférence, Erwan Pépiot reprend le sujet de sa thèse et montre que, parmi un corpus de locuteurices francophones parisien·nes ainsi que dans un corpus de locuteurices anglophones du nord-est des États-Unis, les différences intergenres sur des paramètres prosodiques comme la \say{répartition temporelle consonne/voyelle}, la \say{durée des énoncés} ou encore le voice onset time sont significatives.\\
      Erwan Pépiot est présenté dans la description de \parencite{Pep20}.\\

\parencite{Gar22}\\
      Ce mémoire de Mona Garczarek traite de l'impact de la fréquence fondamentale et de l'intonation sur la perception du genre avec comme objectif de fournir des données afin de développer des méthodes de féminisation de la voix à destination des femmes transgenres.\\
      Mona Garczarek est étudiante à l'université de Liège.\\

\parencite{Can15} \\
      Cet article de Maria Candea et Cyril Trimaille est, comme son nom l'indique, une introduction à la sociophonétique, une discipline encore très jeune (peu de littérature y est donc rattachée pour le moment) qui traite de l'impact social de la phonétique.\\
      Maria Candéa est sociolinguiste et sociophonéticienne à l'Université Paris 3 - Sorbonne Nouvelle. Cyril Trimaille est quant à lui sociolinguiste à l'Université Grenoble Alpes.\\
      Cet article ne s'intègre pas en tant que tel dans le sujet que j'ai choisi, il m'a en revanche été utile pour avoir un aperçu de l'état de l'art de cette discipline (dans laquelle s'intègre mon sujet) ainsi que pour trouver l'article \parencite{Pep20}.\\
      
\parencite{DiC13}\\
      Ce chapitre d'ouvrage traite de la \say{matérialité acoustique et auditive de la prosodie}, il permet d'expliquer, sous l'angle de la physique acoustique, différents paramètres prosodiques comme la fréquence fondamentale, les formants de voyelles, le timbre.\\
      L'auteur de cet ouvrage, Albert Di Cristo, est linguiste à l'université Aix-Marseille. Il concentre ses travaux sur la prosodie.\\
      Ce chapitre d'ouvrage, au même titre que l'article \parencite{Can15}, ne s'intègre pas directement dans le sujet que j'ai choisi mais il m'a aidé à comprendre des notions abordées dans d'autres articles.\\

\printbibliography
