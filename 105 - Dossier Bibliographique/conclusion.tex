 J'ai choisi la thématique à partir d'une question que je me pose depuis plusieurs années : \textit{Comment le genre influence-t-il la voix?}. J'ai voulu aborder cette question sous l'angle de la prosodie de manière à utiliser des paramètres mesurables, et donc, comme je le mentionnais dans l'introduction, trouver des ressources scientifiques sur le sujet. %XXX peut être revenir sur la raison de mon choix
 Lors du premier cours, j'ai trouvé deux sources - \parencite{Can15}, un article présentant l'état de l'art de la sociophonétique et \parencite{Lek16}, un mémoire d'orthophoniste portant sur les différences inter-genres de prosodie - la première m'a permis de trouver le nom d'Aron Arnold, puis de trouver la source \parencite{Pep20}; quant à la seconde, je n'ai malheureusement pas pu y avoir accès, elle ne figure par conséquent pas dans les résumés des ressources bibliographiques. 
 J'ai par la suite emprunté l'ouvrage \parencite{DiC13} afin d'avoir une meilleure compréhension de ce qu'est la prosodie. Dans le même objectif, j'ai regardé une vidéo portant sur la différence entre hauteur et timbre que m'a transmise mon enseignante de TD. 
 Dans le même temps, j'ai emprunté la revue dans laquelle se trouve l'article \parencite{Boe75}, que j'ai trouvé dans la bibliographie de \parencite{Pep20}.
 Toujours dans la bibliographie de \parencite{Pep20}, j'ai trouvé l'acte de conférence \parencite{Pep16}.
 En ce qui concerne l'article de presse grand public, j'ai utilisé l'article \parencite{Bro18}, qui m'a été recommandé par l'enseignante de TD.
 N'ayant pas de nouvelles du mémoire \parencite{Lek16}, j'ai cherché une autre ressource similaire afin de montrer l'influence des paramètres prosodiques sur la perception du genre, les autres sources portant plutôt sur l'influence du genre sur les paramètres prosodiques et la voix de manière plus générale. 
 J'ai finalement trouvé un autre mémoire sur un sujet similaire : \parencite{Gar22}.
 Concernant la source multimédia, j'ai d'abord trouvé le podcast \parencite{podcast}, mais après l'avoir écouté, je ne l'ai pas trouvé pertinent par rapport à la problématique.
 J'ai donc intégré cette vidéo \parencite{video} à la place.\\

 Pour conclure, ce travail m'a appris que le genre a effectivement un impact sur la voix et en particulier sur la prosodie. Par ailleurs, certains paramètres prosodiques comme l'intonation ou la fréquence fondamentale permettent de genrer un·e locuteurice.\\

