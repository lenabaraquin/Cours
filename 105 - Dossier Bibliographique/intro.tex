Pour ce dossier, j'ai décidé de m'intéresser aux liens qu'il existe entre le genre et la prosodie avec la problématique \say{Quel est l'impact du genre, en tant que construit social, sur les paramètres mesurables de la parole?}. \\
J'ai choisi ce thème car je trouve les thématiques liées au genre extrêmement intéressantes.
En effet, dans notre société, le genre infuse dans toutes les interactions interindividuelles, et en particulier, dans celles dont le vecteur est la parole. \\
Je pense qu'étudier l'implication du genre dans la parole au travers de la prosodie permet d'utiliser des paramètres mesurables (comme la fréquence fondamentale, les formants de voyelles, le voice onset time,...) pour répondre à la problématique que j'ai posée, et donc de trouver des ressources scientifiques pour y parvenir.\\

Ce dossier se compose donc de huit sources en lien avec la problématique présentée ci-dessus.\\
Premièrement, un article scientifique qui fait l'hypothèse que \say{bilingual speakers
would adapt their vocal practices to the gender norms of the language they were
using.} : \parencite{Pep20} dont j'ai fait un résumé dans la partie \hyperref[sec:resume]{Résumé de l'article \parencite{Pep20}}.\\
En lien avec ce premier article, j'ai aussi intégré une conférence de \textsc{Pépiot} : \parencite{Pep16}; ainsi qu'une interview d'\textsc{Arnold} retranscrite dans un article de presse : \parencite{Bro18}.\\
Parallèlement à cette première source, j'ai choisi un autre article scientifique \parencite{Boe75}. Dans cet article, les auteurs se proposent de \say{présenter un certain nombre de résultats et d'applications pour l'analyse et la synthèse des faits prosodiques du français} en se basant sur un corpus de 30 hommes et 30 femmes.\\
J'ajoute à ce dossier le mémoire \parencite{Gar22} qui permet de montrer que des paramètres prosodiques comme $F_0$ ou les modèles d'intonations ont un impact dans la manière dont sera genré·e un·e locuteurice.\\
S'agissant de la source multimédia, j'ai choisi une vidéo de la chaine youtube TransVoiceLessons : \parencite{video} qui traite de l'impact de la modification de la fréquence de résonance du larynx sur la perception du genre vocal.
Enfin, deux autres sources m'ont permis de comprendre le sujet. Un article scientifique qui reprend l'état de l'art de la sociophonétique : \parencite{Can15} ainsi qu'un chapitre d'ouvrage scientifique qui explique les critères physiques de certains paramètres prosodiques : \parencite{DiC13}.\\
