Pour ce dossier, j'ai décidé de m'intéresser aux liens qu'il existe entre le genre et la prosodie avec la problématique XXX. \\
J'ai choisi ce thème car je trouve les thématiques liées aux genre extrêmement interessantes.
En effet, dans notre société, le genre infuse dans toutes les interactions interindividuelles, et en particulier, dans celles dont le vecteur est la parole. \\
Je pense qu'étudier l'implication du genre dans la parole au travers de la prosodie permet d'utiliser des paramètres mesurables (comme la fréquence fondammentale, les formants de voyelles, le voice onset time,...) pour répondre à la problématique que j'ai posée, et donc de trouver des ressources scientifiques pour y parvenir.\\

Ce dossier se compose donc de huit sources en lien avec la problématique présentée ci-dessus.\\
Premièrement, un article scientifique qui fait l'hypothèse que \say{XXX} : \cite{Pep20} dont j'ai fait un résumé dans la partie ;;; faire un lien vers la partie en question ;;;.\\
En lien avec ce premier article, j'ai aussi intégré une conférence de \textsc{Pépiot} : \cite{Pep16}; ainsi qu'une interview d'\textsc{Arnold} retranscrite dans un article de presse : \cite{Bro18}.\\
Parallèlement à cette première source, j'ai choisi un autre article scientifique qui cette fois fait l'hypothèse que \say{XXX} : \cite{Boe75}.\\
PODCAST\\
J'ajoute à ce dossier le mémoire \cite{Gar22} qui permet de montrer que des paramètres prosodiques comme $F_0$ ou les modèles d'intonations ont un impact dans la manière dont sera genrer un·e locuteurice.\\
Enfin, deux autres sources m'ont permis de comprendre le sujet. Un article scientifique qui reprend l'état de l'art de la sociophonétique : \cite{Can18} ainsi qu'un chapitre d'ouvrage scientifique qui explique les critères physiques de certains paramètres prosodiques : \cite{DiC13}.\\
