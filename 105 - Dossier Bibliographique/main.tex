\documentclass[a4paper,times,12pt]{article}
\usepackage[utf8]{inputenc}
\usepackage[T1]{fontenc}
\usepackage[french]{babel}
\usepackage{graphicx}
\usepackage{fullpage}
\usepackage{eso-pic}
\usepackage[left = \flqq{}, right = \frqq{}]{dirtytalk}%citations avec \say
\usepackage{hyperref}
\usepackage[style=apa,sortcites=true,backend=biber]{biblatex}
%biblatex:
\DeclareLanguageMapping{french}{french-apa}
\addbibresource{biblio.bib}

\newcommand{\HRule}{\rule{\linewidth}{0.5mm}}
\newcommand{\blap}[1]{\vbox to 0pt{#1\vss}}
\newcommand\AtUpperLeftCorner[3]{%
  \put(\LenToUnit{#1},\LenToUnit{\dimexpr\paperheight-#2}){\blap{#3}}%
}
\newcommand\AtUpperRightCorner[3]{%
  \put(\LenToUnit{\dimexpr\paperwidth-#1},\LenToUnit{\dimexpr\paperheight-#2}){\blap{\llap{#3}}}%
}
 
\title{\LARGE{Genre et prosodie}\\·\\\textit{Quel est l'impact du genre, en tant que construit social, sur les paramètres mesurables de la parole?}}
\author{\textsc{Baraquin} Léna\\22208376}
\date{\today}
\makeatletter
 
\begin{document}
 
\begin{titlepage}
    \enlargethispage{2cm}
 
    \AddToShipoutPicture{
        \AtUpperLeftCorner{2.1cm}{1cm}{\includegraphics[width=4cm]{logo_fac.jpg}}
    }
\phantom{}\\
Licence SDL 1\up{ère} année\\
UE SL00105T\\
Enseignante : Laure Fesquet\\
\@author 
    \begin{center}
        \vspace*{8cm}
 
        \textsc{\@title}
        \HRule
        \vspace*{0.5cm}
 
    \end{center}
 
    \vspace*{9.2cm}
Année universitaire 2022/2023 
\end{titlepage}
\ClearShipoutPicture

\tableofcontents
\section{Introduction}
Pour ce dossier, j'ai décidé de m'intéresser aux liens qu'il existe entre le genre et la prosodie avec la problématique : \textit{Quel est l'impact du genre, en tant que construit social, sur les paramètres mesurables de la parole?} \\
J'ai choisi ce thème car je trouve les thématiques liées au genre extrêmement intéressantes.
En effet, dans notre société, le genre infuse dans toutes les interactions interindividuelles, et en particulier, dans celles dont le vecteur est la parole. \\
Je pense qu'étudier l'implication du genre dans la parole au travers de la prosodie permet d'utiliser des paramètres mesurables (comme la fréquence fondamentale, les formants de voyelles, le voice onset time,...) pour répondre à la problématique que j'ai posée et donc de trouver des ressources scientifiques pour y parvenir.\\

Ce dossier se compose de huit sources en lien avec la problématique présentée ci-dessus.\\
Premièrement, un article scientifique qui fait l'hypothèse que \say{bilingual speakers
would adapt their vocal practices to the gender norms of the language they were
using. } : (\cite{Pep20}, p.156) dont j'ai fait un résumé dans la partie \hyperref[sec:resume]{Résumé de l'article \parencite{Pep20}}.\\
En lien avec ce premier article, j'ai aussi intégré une conférence de \textsc{Pépiot} : \parencite{Pep16}; ainsi qu'une interview d'\textsc{Arnold} retranscrite dans un article de presse : \parencite{Bro18}.\\
Parallèlement à cette première source, j'ai choisi un autre article scientifique \parencite{Boe75}. Dans cet article, les auteurs se proposent de \say{présenter un certain nombre de résultats et d'applications pour l'analyse et la synthèse des faits prosodiques du français } en se basant sur un corpus de 30 hommes et 30 femmes.\\
J'ajoute à ce dossier le mémoire \parencite{Gar22} qui permet de montrer que des paramètres prosodiques comme $F_0$ ou les modèles d'intonations ont un impact dans la manière dont sera genré·e un·e locuteurice.\\
Concernant la source multimédia, j'ai choisi une vidéo de la chaine youtube TransVoiceLes- sons : \parencite{video} qui traite de l'impact de la modification de la fréquence de résonance du larynx sur la perception du genre vocal.
Enfin, deux autres sources m'ont permis de mieux comprendre le sujet. Un article scientifique qui reprend l'état de l'art de la sociophonétique : \parencite{Can15} ainsi qu'un chapitre d'ouvrage scientifique qui explique les critères physiques de certains paramètres prosodiques : \parencite{DiC13}.



\newpage


%\section{Résumé de l'article \parencite{Pep20}}
\label{sec:resume}
%Article de Erwan Pépiot et Aron Arnold, respectivement chercheurs en XXX
Introduction de l'article XXX
Hypothèse de l'article XXX
Rappel de l'état de l'art XXX
- traits prosodiques sous l'angle de l'anatomie
- tout ne se base pas sur l'anatomie (aussi composante sociale "it is also the result of a culturally gendered performance" (p155))
- différences inter-genre qui varient selon la langue (Peut etre donner des exemples (paragraphe P4))
- modulation (paramètre prosodique)
- type de phonation (GOQ - H1-H2)
   notamment Gordon et Ladefoged 2001
- VOT
Le sujet de cette étude est inédit.(incister sur les détails de l'étude)
Déscription de l'étude :
   Porte sur des locuteurs bilingues anglais - français en mesurant plusieurs paramètres :
      - $F_0$, formants de voyelles, VOT, H1-H2; en discours spontané et en discours lu; dans les deux langues
Hypothèse de l'article : 
"Bilingual speakers would adapt their vocal practices to the gender norms of the language they were using"


\section{Bibliographie}
\cite{Pep20} \\
      Cet article scientifique d'Erwan Pépiot et d'Aron Arnold porte sur les différences de prosodie dans un corpus de locuteurices bilingues anglais - français en fonction de leur genre.\\
      Erwan Pépiot est chercheur en linguistique à l'université Paris 8 et Aron Arnold est chercheur en études de genre et en sociophonétique à l'université catholique de Louvain \\

\cite{Bro18} \\
      Cet article de presse d'Émilie Brouze est une interview de Aron Arnold (déja présenté dans \cite{Pep20}) à propos de l'implication du genre dans la voix.\\
      Émilie Brouze est autrice et journaliste à l'Obs.

\cite{podcast} \\
      Ce podcast traite du genre dans la voix chantée. Il fait intervenir deux chanteurs ainsi qu'un enseignant de littérature pour XXX; respectivement Gérard Lesne, Artur H et Jérôme Solal.
      Ce podcast est produit par Jérôme Sandlarz, producteur chez radio france et réalisé par Somany Na, Vincent Abouchar, Olivier Bétard et Hervé Marchon, réalisateurices chez radio france.

\cite{video} \\
      Cette video de Amelia Huff traite 
      Amelia Huff, publiquement connue sous le nom de Zhea Erose, est une vidéaste et musicienne.

\cite{Boe75} \\
      Cet article scientifique de Louis-Jean Boë, Michel Contini et Hippolyte Rakotofiringa traite de XXX\\
      Louis-Jean Boë, Michel Contini et Hippolyte Rakotofiringa sont tous trois chercheur en sciences de la parole à l'université de Grenoble Alpes.\\

\cite{Pep16} \\
      Dans cette conférence, Erwan Pépiot montre, en se basant sur sa thèse, que ...\\
      Erwan Pépiot est présenté dans la description de \cite{Pep20}.\\

\cite{Can15} \\
      Cet article de Maria Candea et Cyrill Trimaille est, comme son nom l'indique, une introduction à la sociophonétique, une discipline encore très jeune (peu de littérature y est donc rattaché pour le moment) qui traite de l'impact social de la phonétique.\\
      ; présentation ;
      Cet article ne s'intègre pas en tant que tel dans le sujet que j'ai choisi, il m'a en revanche été utile pour avoir un aperçu de l'état de l'art de cette discipline (dans laquelle s'integre mon sujet) ainsi que pour trouver l'article \cite{Pep20}.\\
      
\cite{DiC13}\\
      Ce chapitre d'ouvrage traite de la \say{matérialité acoustique et auditive de la prosodie}, il permet d'expliquer, sous l'angle de la physique ;;;accoustique;;;, différents paramètres prosodiques comme la fréquence fondammentale, les formants de voyelles, le timbre.\\
      ;;; présentation ;;;
      Ce chapitre d'ouvrage, au même titre que l'article \cite{Can15}, ne s'intègre pas directement dans le sujet que j'ai choisi mais il m'a aidé à comprendre des notions abordées dans d'autres articles.\\

\cite{Gar22}\\
      Ce mémoire traite de \\
      Mona Garczarek est \\

\printbibliography

\section{Analyse réflexive et Conclusion}
 J'ai choisi la thématique à partir d'une question que je me pose depuis plusieurs années : \textit{Comment le genre influence-t-il la voix?}. J'ai voulu aborder cette question sous l'angle de la prosodie de manière à utiliser des paramètres mesurables, et donc, comme je le mentionnais dans l'introduction, trouver des ressources scientifiques sur le sujet. %XXX peut être revenir sur la raison de mon choix
 Lors du premier cours, j'ai trouvé deux sources - \parencite{Can15}, un article présentant l'état de l'art de la sociophonétique et \parencite{Lek16}, un mémoire d'orthophoniste portant sur les différences inter-genres de prosodie - la première m'a permis de trouver le nom d'Aron Arnold, puis de trouver la source \parencite{Pep20}; quant à la seconde, je n'ai malheureusement pas pu y avoir accès, elle ne figure par conséquent pas dans les résumés des ressources bibliographiques. 
 J'ai par la suite emprunté l'ouvrage \parencite{DiC13} afin d'avoir une meilleure compréhension de ce qu'est la prosodie. Dans le même objectif, j'ai regardé une vidéo portant sur la différence entre hauteur et timbre que m'a transmise l'enseignante de TD. 
 Dans le même temps, j'ai emprunté la revue dans laquelle se trouve l'article \parencite{Boe75}, que j'ai trouvé dans la bibliographie de \parencite{Pep20}.
 Toujours dans la bibliographie de \parencite{Pep20}, j'ai trouvé l'acte de conférence \parencite{Pep16}.
 En ce qui concerne l'article de presse grand public, j'ai utilisé l'article \parencite{Bro18}, qui m'a été recommandé par l'enseignante de TD.
 N'ayant pas de nouvelles du mémoire \parencite{Lek16}, j'ai cherché une autre ressource similaire afin de montrer l'influence des paramètres prosodiques sur la perception du genre, les autres sources portant plutôt sur l'influence du genre sur les paramètres prosodiques et la voix de manière plus générale; j'ai finalement trouvé un autre mémoire sur un sujet similaire : \parencite{Gar22}.
 S'agissant de la source multimédia, j'ai d'abord trouvé le podcast \parencite{podcast}, mais après l'avoir écouté, je ne l'ai pas trouvé pertinent par rapport à la problématique; j'ai donc intégré cette vidéo \parencite{video} à la place.\\

 Pour conclure, ce travail m'a appris que le genre a effectivement un impact sur la voix et en particulier sur la prosodie. Par ailleurs, certains paramètres prosodiques comme l'intonation ou la fréquence fondamentale permettent de genrer un·e locuteurices.\\




\end{document}




































\section{Questions à poser}
\begin{itemize}
         \item Comment citer un podcast? (en particulier qui sont les auteurs?)
         \item Comment citer un article de journal?                            \begin{itemize}
         \item Comment compiler proprement la bibliographie?\\                    \item https://www.nouvelobs.com/rue89/nos-vies-intimes/20180723.OBS0077/grave-ou-aigue-sombre-ou-claire-ce-que-le-genre-fait-a-notre-voix.html
            Utiliser la commande biber?                                           \item https://matheo.uliege.be/handle/2268.2/2191
         \item CM du 16 novembre!!!                                               \item https://www.cairn.info/revue-langage-et-societe-2015-1-page-7.html
\end{itemize}                                                                  \end{itemize}

Problématique :\\
Comment le genre d'un.e locuteur.ice du français influe sur sa prosodie?

Déroulement session 1 vendredi 7/10
lecture de https://www.cairn.info/revue-langage-et-societe-2015-1-page-7.html
   \enquote{Les aspects novateurs concernant les dynamiques du changement linguistique (attrition, nivellement) ou les patterns de variation genrée ont été « redécouverts » tout récemment.}
   repère 34
   \enquote{thèses de doctorat, notamment celle de Fagyal (1995) sur le style vocal de Marguerite Duras selon les situations}
   repère 36
   \enquote{Concernant le français, le courant cognitif est surtout représenté par des études en acquisition (cf. Nardy, Chevrot et Barbu dans ce numéro) mais nous pouvons également mentionner la thèse d’Aubanel (2011) qui cherche à articuler expérimentalement étude de la variation phonétique diatopique, interaction conversationnelle, traitement automatique et « caractérisation des représentations mentales associées aux sons de la parole ». }
   TROP INTERESSANT, repère 38
établissement d'une problématique
 
\cite{tes1}
bla bla \parencite{Pep21} bla bla bla
 

biblio.bib

Page de garde :

    nom de l'université ou logo,
    nom et code de l'UE,
    année universitaire,
    formation (licence  SDL  première  année et mineure de découverte)
    nom de l'enseignant et groupe de TD / ou SED
    votre  nom  et  prénom
    votre numéro d'étudiant
    titre = votre thématique
    sous-titre = votre problématique
    image illustrant la thématique

