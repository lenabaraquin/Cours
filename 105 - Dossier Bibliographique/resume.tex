Article de Erwan Pépiot et Aron Arnold, respectivement chercheurs en XXX
Introduction de l'article XXX
Hypothèse de l'article XXX
Rappel de l'état de l'art XXX
- traits prosodiques sous l'angle de l'anatomie
- tout ne se base pas sur l'anatomie (aussi composante sociale "it is also the result of a culturally gendered performance" (p155))
- différences inter-genre qui varient selon la langue (Peut etre donner des exemples (paragraphe P4))
- modulation (paramètre prosodique)
- type de phonation (GOQ - H1-H2)
   notamment Gordon et Ladefoged 2001
- VOT
Le sujet de cette étude est inédit.(incister sur les détails de l'étude)
Déscription de l'étude :
   Porte sur des locuteurs bilingues anglais - français en mesurant plusieurs paramètres :
      - $F_0$, formants de voyelles, VOT, H1-H2; en discours spontané et en discours lu; dans les deux langues
Hypothèse de l'article : 
"Bilingual speakers would adapt their vocal practices to the gender norms of the language they were using"

 Les auteurs font l'hypothèse que les paramètre prosodiques qui permettent de genrer un individu ne dépendent pas seulement de données anatomiques et physiologique mais aussi de données culturelles. Pour montrer cette hypothèse,??? les auteurs font appel à un corpus de XXX bilingues XXX pour montrer que 
(plutot traite de ...)montre l'implication des normes culturelles de genre dans les différences des paramètres prosodiques de la voix des locuteurs en fonction de leur genre au travers d'une étude sur un corpus de locuteurices bilingues anglais - français.
