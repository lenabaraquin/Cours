Cet article comporte pour commencer un résumé de l'état de l'art dans lequel les auteurs reprennent les études faites sur le sujet des différences intergenre dans la voix dans plusieurs langues.
Ce rappel permet notamment de revenir sur de précédentes études portant sur les différences anatomiques sexuées : \say{[The] acoustic gender differences are partly due to developmental differences in the human vocal appartus that emerge during puberty } (\cite{Pep20}, p.154).
Les auteurs rappellent par ailleurs que \say{[the voice] is also the result of a culturally gendered performance } (\cite{Pep20}, p.155). 
Le reste de ce rappel de l'état de l'art présente des articles s'appuyant sur des paramètres prosodiques comme la fréquence fondamentale, la modulation de celle-ci, ou encore le \textit{voice onset time} pour en déduire des différences intergenres.\\
Les auteurs posent ensuite l'hypothèse suivante : \say{[...] biblingual speakers would adapt their vocal practices to the gender norms of the langage they were using } (\cite{Pep20}, p.156).
Pour vérifier cette hypothèse, l'étude se base sur 6 locutrices et 6 locuteurs bilingues anglais français.\\
En s'appuyant sur les paramètres prosodiques suivants : $F_0$ moyenne, variations de $F_0$, formants de voyelles, \textit{voice onset time}, différence d'intensité entre la première et la seconde harmonique, l'étude montre non seulement qu'il existe des différences intergenres sur ces paramètres mais aussi que ces différences ne sont pas les mêmes en anglais et en français.\\
%Une des limites posée par les auteurs est le nombre réduit de participant·e·s à cette étude.
%Les auteurs posent aussi comme limite que les participants sont des \say{native English speakers } (\cite{Pep20}, p.174).\\
Cependant, les auteurs posent comme limites que le nombre de participant·e·s à cette étude est trop réduit et que la langue maternelle des participant·e·s est l'anglais.\\

Les auteurs concluent que \say{the process by which female or male voices are produced involves different vocal practices, and these practices vary from one language to another } (\cite{Pep20}, p.174).\\
